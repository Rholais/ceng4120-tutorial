%!TEX program = xelatex
\documentclass[10pt, compress]{beamer}
\usepackage[titleprogressbar]{../../cls/beamerthemem}

\setbeamertemplate{caption}[numbered]
\setbeamertemplate{theorems}[numbered]
\newcounter{example}
\resetcounteronoverlays{example}
\newtheorem{crl}{Corollary}[theorem]
\newtheorem{eg}[example]{Example}
\newtheorem*{solution*}{Solution}

\usepackage{booktabs}
\usepackage[scale=2]{ccicons}
\usepackage{minted}

\usepackage{cleveref}
\crefname{example}{Example}{Examples}

\usepgfplotslibrary{dateplot}

\usemintedstyle{trac}

\usepackage{algorithm}
\usepackage[noend]{algpseudocode}
\resetcounteronoverlays{algorithm}

\usepackage{listings}
\usepackage{mathtools}
\usepackage{multicol}
\usepackage{qtree}

\usepackage{tikz}

\usepackage{version}
% \excludeversion{proof}
% \excludeversion{solution*}

\makeatletter
\def\old@comma{,}
\catcode`\,=13
\def,{%
    \ifmmode%
    \old@comma\discretionary{}{}{}%
    \else%
    \old@comma%
    \fi%
}
\makeatother

\title{CENG 4120 Tutorial of Week 04}
\subtitle{Verilog Hardware Description Language}
\author{LI Haocheng}
\institute{Department of Computer Science and Engineering}

\begin{document}

\maketitle

\begin{frame}
\frametitle{Hardware Description Language}
\begin{itemize}
    \item Text of expressions, statements and control structures
    \item Structure and behavior of electronic circuits
    \item High level without being tied to a specific electronic technology
    \item Use \begin{itemize}
        \item Model a piece of hardware before it is created physically
        \item Validate the specification against the HDL implementation
        \item Produce an equivalent netlist of generic hardware primitives
    \end{itemize}
    \item VHDL and Verilog
\end{itemize}
\end{frame}

\begin{frame}
\frametitle{Verilog HDL}
\begin{itemize}
    \item Gateway Design Automation developed Verilog-XL in 1984
    \item Acquired and put into public domain by Cadence Design Systems
    \item IEEE Standard 1364-[1995, 2001, 2005], 1800-[2009, 2017]
\end{itemize}
\end{frame}

\begin{frame}[fragile]
\frametitle{A Simple Design}
\begin{columns}
    \column{.6\linewidth}
    \lstset{language=Verilog,
        basicstyle=\ttfamily\scriptsize
    }
\begin{lstlisting}
//  Verilog code for a full adder
module FAX1 (YC, B, C, A, YS);
    input  B ;
    input  C ;
    input  A ;
    output YC ;
    output YS ;

    assign YC = (A & B) | (A & C) | (B & C);
    assign YS = A ^ B ^ C;

endmodule
\end{lstlisting}

\column{.5\linewidth}
Comments describe or document models
\lstset{language=Verilog,
    basicstyle=\ttfamily\scriptsize
}
\begin{lstlisting}
// a single line with comments
/* multiple lines
containing comments */
\end{lstlisting}

A module defines interface and behavior
\lstset{language=Verilog,
    basicstyle=\ttfamily\scriptsize
}
\begin{lstlisting}
module identifier (port_list) ;
    ports_declaration;
    module_body;
endmodule
\end{lstlisting}
\end{columns}

\end{frame}

\begin{frame}[fragile]
\frametitle{Port Declaration}
\begin{columns}
    \column{.7\linewidth}
    \lstset{language=Verilog,
        basicstyle=\ttfamily\scriptsize
    }
\begin{lstlisting}
//  Verilog code for a full adder
module FAX1 (output YC, input B, C, A, output YS);
    /*
    input  B ;
    input  C ;
    input  A ;
    output YC ;
    output YS ;
    */

    assign YC = (A & B) | (A & C) | (B & C);
    assign YS = A ^ B ^ C;

endmodule
\end{lstlisting}
    
    \column{.4\linewidth}
    A port defines name and direction.
    
    \lstset{language=Verilog,
        basicstyle=\ttfamily\scriptsize
    }
\begin{lstlisting}
input [3 : 0] a, b ;
output [3 : 0] c;
\end{lstlisting}
    
    
    Ports can be defined in port list.
    
    Continuous assignments model combinational logic.
\end{columns}

\end{frame}
\begin{frame}[fragile]
\frametitle{Wire Declaration}
\begin{columns}
    \column{.7\linewidth}
    \lstset{language=Verilog,
        basicstyle=\ttfamily\scriptsize
    }
\begin{lstlisting}
//  Verilog code for a full adder
module FAX1 (output YC, input B, C, A, output YS);

    wire AB, AC, BC;
    
    assign AB = A & B;
    assign AC = A & C;
    assign BC = B & C;

    assign YC = AB | AC | BC;
    assign YS = A ^ B ^ C;

endmodule
\end{lstlisting}
    
    \column{.4\linewidth}
    A wire defines an internal signal.
    
    These statements are independent and executed concurrently.
\end{columns}

\end{frame}

\begin{frame}[fragile]
\frametitle{Wire Assignment}
\begin{columns}
\column{.7\linewidth}
\lstset{language=Verilog,
    basicstyle=\ttfamily\scriptsize
}
\begin{lstlisting}
//  Verilog code for a full adder
module FAX1 (output YC, input B, C, A, output YS);

    wire AB = A & B;
    wire AC = A & C;
    wire BC = B & C;
    
    assign YC = AB | AC | BC;
    assign YS = A ^ B ^ C;

endmodule
\end{lstlisting}

\column{.4\linewidth}
A wire can be declared and continuously assigned in a single statement.
\end{columns}

\end{frame}

\begin{frame}[fragile]
\frametitle{Module Instantiation}
\begin{columns}
    \column{.7\linewidth}
    \lstset{language=Verilog,
        basicstyle=\ttfamily\scriptsize
    }
\begin{lstlisting}
//  Verilog code for an and gate
module ANDX1 (output Y, input A, B);
    assign Y = A & B;
end module

//  Verilog code for a full adder
module FAX1 (output YC, input B, C, A, output YS);
    wire AB = A & B;
    ANDX1 A1 (AC, A, C);
    ANDX1 A2 (.Y(BC), .A(B), .B(C));
    
    assign YC = AB | AC | BC;
    assign YS = A ^ B ^ C;
endmodule
\end{lstlisting}
    
    \column{.4\linewidth}
    Connections by order must match the order appearing in the instantiated module.
    
    Connections by name can be specified in a set of parenthesis following the name of ports.
    \lstset{language=Verilog,
        basicstyle=\ttfamily\scriptsize
    }
\begin{lstlisting}
ANDX1 A2(.Y(BC), .A(B), .B(C));
ANDX1 A3(.A(B), .B(C), .Y(BC));
\end{lstlisting}

Output pins can be left unconnected.
\begin{lstlisting}
ANDX1 A4(, B, C);
\end{lstlisting}

\end{columns}

\end{frame}

\begin{frame}[fragile]
\frametitle{Vectors}
\begin{columns}
    \column{.7\linewidth}
    \lstset{language=Verilog,
        basicstyle=\ttfamily\scriptsize
    }
    \begin{lstlisting}
//  Verilog code for an and gate
module ANDX1 (output Y, input A, B);
    assign Y = A & B;
end module

//  Verilog code for an and gate array
module ANDX3 (output [2:0] Y, input [2:0] A, B);
    ANDX1 A1 [2:0] (Y, A, B);
end module

//  Verilog code for a full adder
module FAX1 (output YC, input B, C, A, output YS);
    ANDX3 A1 ({AB, AC, BC}, {A, A, B}, {B, C, C});

    assign YC = AB | AC | BC;
    assign YS = A ^ B ^ C;
endmodule
    \end{lstlisting}
    
    \column{.4\linewidth}
    Module instantiations can create an array of instances.
    
    The concatenation is the combination of two or more expressions.
    \lstset{language=Verilog,
        basicstyle=\ttfamily\scriptsize
    }
\begin{lstlisting}
wire [7:0] a;
{a, 4'b1110, b[2:1]}
\end{lstlisting}

\end{columns}

\end{frame}

\begin{frame}[fragile]
\frametitle{Repetition Multiplier}
\begin{columns}
    \column{.7\linewidth}
    \lstset{language=Verilog,
        basicstyle=\ttfamily\scriptsize
    }
\begin{lstlisting}
//  Verilog code for an and gate
module ANDX1 (output Y, input A, B);
    assign Y = A & B;
end module

//  Verilog code for an and gate array
module ANDX3 (output [2:0] Y, input [2:0] A, B);
    ANDX1 A1 [2:0] (Y, A, B);
end module

//  Verilog code for a full adder
module FAX1 (output YC, input B, C, A, output YS);
    ANDX3 A1 ({AB, AC, BC}, {{2{A}}, B}, {B, {2{C}}});
    
    assign YC = AB | AC | BC;
    assign YS = A ^ B ^ C;
endmodule
\end{lstlisting}
    
    \column{.4\linewidth}
    Concatenations can also be expressed using a repetition multiplier.
    \lstset{language=Verilog,
        basicstyle=\ttfamily\scriptsize
    }
\begin{lstlisting}
{a, {2{b, c, d}}, a}
{a, b, c, d, b, c, d, a}
\end{lstlisting}

\end{columns}

\end{frame}

\plain{Questions?}

\end{document}
