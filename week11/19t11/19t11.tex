%!TEX program = xelatex
\documentclass[10pt, compress, handout]{beamer}
\usepackage[titleprogressbar]{../../cls/beamerthemem}

\setbeamertemplate{caption}[numbered]
\setbeamertemplate{theorems}[numbered]
\newcounter{example}
\resetcounteronoverlays{example}
\newtheorem{crl}{Corollary}[theorem]
\newtheorem{eg}[example]{Example}
\newtheorem*{solution*}{Solution}

\usepackage{booktabs}
\usepackage[scale=2]{ccicons}
\usepackage{minted}

\usepackage{cleveref}
\crefname{example}{Example}{Examples}

\usepgfplotslibrary{dateplot}

\usemintedstyle{trac}

\usepackage{algorithm}
\usepackage[noend]{algpseudocode}
\resetcounteronoverlays{algorithm}

\usepackage[linguistics]{forest}
\usepackage{listings}
\usepackage{mathtools}
\usepackage{multicol}
\usepackage{qtree}

\usepackage{tikz}

\usepackage{version}
% \excludeversion{proof}
% \excludeversion{solution*}

\makeatletter
\def\old@comma{,}
\catcode`\,=13
\def,{%
    \ifmmode%
    \old@comma\discretionary{}{}{}%
    \else%
    \old@comma%
    \fi%
}
\makeatother

\title{CENG 4120 Tutorial of Week 11}
\subtitle{Rectlinear Minimum Spanning Tree}
\author{LI Haocheng}
\institute{Department of Computer Science and Engineering}

\begin{document}

\maketitle

\begin{frame}
\frametitle{RMST}
\begin{definition}
    A \textbf{tree} is a graph whose $n$ nodes are connected by $n - 1$ edges.
\end{definition}
\begin{definition}
    A \textbf{spanning tree} in a graph $G(V,E)$ is a connected, acyclic subgraph $G'$ contained
    within $G$ that includes (spans) every node $v \in V$.
\end{definition}
\begin{definition}
    A \textbf{minimum spanning tree} (MST) is a spanning tree with the smallest possible sum
    of edge costs.
\end{definition}
\begin{definition}
    Given a set of points $V$ , an undirected
    graph $G(V , E)$ is called a \textbf{spanning graph} if it
    contains a minimum spanning tree.
\end{definition}
\begin{definition}
    A \textbf{rectilinear minimum spanning tree} (RMST) is an MST where all edge lengths
    correspond to the Manhattan (rectilinear) distance metric.
\end{definition}
\end{frame}

\begin{frame}
\frametitle{Minimum Spanning Tree}
\begin{eg}
    Construct RMST with $O(|E| \log |V|)$ time complexity.
\end{eg}
\onslide<2->\begin{solution*}
    Use Kruskal's Algorithm.
\end{solution*}
\onslide<1->\begin{theorem}
    For any cycle $C$ in the graph, if the weight of an edge $e$ of $C$ is larger than the individual weights of all other edges of $C$, then this edge cannot belong to an MST.
\end{theorem}
\onslide<3>\begin{proof}
    Assume the contrary, i.e. that $e$ belongs to an MST $T_1$. Then deleting $e$ will break $T_1$ into two sub-trees with the two ends of $e$ in different sub-trees. The remainder of $C$ reconnects the sub-trees, hence there is an edge $f$ of $C$ with ends in different sub-trees, i.e., it reconnects the sub-trees into a tree $T_2$ with weight less than that of $T_1$.
\end{proof}
\end{frame}

\begin{frame}[fragile]
\frametitle{Uniqueness Property}
\begin{columns}
    \column{.5\linewidth}
    \onslide<1->\begin{definition}
        Given a point $s$, a region $R$ has the
        \textbf{uniqueness property} with respect to $s$ if for every
        pair of points $p, q \in R$, $\|pq\|_1 < max(\|sp\|_1, \|sq\|_1)$.
        A partition of space into a finite set of disjoint regions
        is said to have the uniqueness property if each of its
        regions has the uniqueness property.
    \end{definition}
\onslide<1->\begin{theorem}
    Given a point $s$ in the plane, the octal
    partition with respect to $s$ has the uniqueness property.
\end{theorem}
\column{.6\linewidth}
\onslide<2->\begin{proof}
    Suppose we have two points $p$ and $q$ in $R_1$.
    W.l.o.g, we can assume $x_p \le x_q$ .
    If $y_p \le y_q$,
    then we have $\|sq\|_1 = \|sp\|_1 + \|pq\|_1 > \|pq\|_1$.
    Therefore, we only need to consider the case when $y_p > y_q$.
    In this case, we have \begin{align}
        \begin{aligned}
        pq = & |x_p - x_q| + |y_p - y_q|\\
        = & x_q - x_p + y_p - y_q\\
        = & (x_q - y_q) + y_p - x_p\\
        < & (x_s - y_s) + y_p - x_s\\
        = & y_p - y_s\\
        \le & x_p - x_s + y_p - y_s = \|sp\|_1.
        \end{aligned}
    \end{align}
\end{proof}
\end{columns}
\end{frame}

\begin{frame}[fragile]
\frametitle{Active Set}
\begin{columns}
    \column{.5\linewidth}
    \onslide<1->\begin{definition}
        \textbf{Active set} consisting of points whose nearest neighbors in R1 are
        still to be discovered.
    \end{definition}

\begin{theorem}
    For any two points $p, q$ in the active set,
    we have $x_p \ne x_q$ , and if $x_p < x_q$ then $x_p - y_p \le x_q - y_q$.
\end{theorem}

\onslide<2->\begin{proof}
    Otherwise, $p$ is not active.
\end{proof}

    \column{.5\linewidth}
    \onslide<1->\begin{theorem}
        When the sweep step processes point $s$,
        we find a subset of active points which have $s$ in their $R_1$ regions.
        We need only to connects with the nearest active point.
    \end{theorem}

\onslide<3>\begin{proof}
    W.l.o.g, suppose $p$ and $q$ are two of them.
    Then $p$ and $q$ are in the $R_5$ region of $s$,
    which means $\|pq\|_1 < max(\|sp\|_1, \|sq\|_1)$.
\end{proof}
\end{columns}
\end{frame}

\begin{frame}[fragile]
\frametitle{RMST Construction}
\begin{columns}
    \column{.3\linewidth}
    \onslide<1->\begin{theorem}
        Given $n$ points in the plane, the algorithm constructs a spanning
        graph in $O(n \log n)$ time, and the number of edges in
        the graph is $O(n)$.
    \end{theorem}
    
    \column{.8\linewidth}
    \begin{algorithmic}
        \Procedure{RectilinearSpanningGraph}{}
        \State sort points according to $x + y$;
        \State $A \coloneqq \emptyset$;
        \For{each point $p$ in the order}
        \State find points in $A$ such that $p$ is in their $R_1$ region
        \State connect $p$ with the nearest point in the subset
        \State delete the subsets from $A$
        \State $A \coloneqq A \cap \{p\}$
        \EndFor
        \EndProcedure
    \end{algorithmic}
\end{columns}

\onslide<2>\begin{proof}
    All edges that we delete are redundant based on the cycle
    property. Thus, the output graph of the algorithm will
    contain at least one rectilinear minimum spanning tree.
    In the algorithm, each given point will be inserted
    and deleted at most once from the active set for $R_1$,
    For each insertion or
    deletion, the algorithm requires $O(\log n)$ time.
    Thus,
    the total time is upper bounded by $O(n \log n)$.
\end{proof}
\end{frame}

\plain{Questions?}

\end{document}
