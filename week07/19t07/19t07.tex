%!TEX program = xelatex
\documentclass[10pt, compress, handout]{beamer}
\usepackage[titleprogressbar]{../../cls/beamerthemem}

\setbeamertemplate{caption}[numbered]
\setbeamertemplate{theorems}[numbered]
\newcounter{example}
\resetcounteronoverlays{example}
\newtheorem{crl}{Corollary}[theorem]
\newtheorem{eg}[example]{Example}
\newtheorem*{solution*}{Solution}

\usepackage{booktabs}
\usepackage[scale=2]{ccicons}
\usepackage{minted}

\usepackage{cleveref}
\crefname{example}{Example}{Examples}

\usepgfplotslibrary{dateplot}

\usemintedstyle{trac}

\usepackage{algorithm}
\usepackage[noend]{algpseudocode}
\resetcounteronoverlays{algorithm}

\usepackage{listings}
\usepackage{mathtools}
\usepackage{multicol}
\usepackage{qtree}

\usepackage{tikz}

\usepackage{version}
% \excludeversion{proof}
% \excludeversion{solution*}

\makeatletter
\def\old@comma{,}
\catcode`\,=13
\def,{%
    \ifmmode%
    \old@comma\discretionary{}{}{}%
    \else%
    \old@comma%
    \fi%
}
\makeatother

\title{CENG 4120 Tutorial of Week 07}
\subtitle{Logic Minimization}
\author{LI Haocheng}
\institute{Department of Computer Science and Engineering}

\begin{document}

\maketitle

\begin{frame}
\frametitle{Tautology}
\begin{eg}
    What are the termination rules for tautology?
\end{eg}
\onslide<2>\begin{solution*}
    \begin{itemize}
        \item A cover is a tautology when it has a row of all $1$s.
        \item A cover is not a tautology when it has a column of $0$s.
        \item A cover is a tautology when it depends on one variable only and there is no
        column of $0$s in that field.
        \item A cover is not a tautology when it is unate and there is not a row of all
        $1$s.
    \end{itemize}
\end{solution*}
\end{frame}

\begin{frame}[fragile]
\frametitle{Complementation}
\begin{columns}
    \column{\linewidth}
    \begin{eg}
        What are the termination rules for complementation?
    \end{eg}
    \onslide<2>\begin{solution*}
        \begin{itemize}
            \item $f = 0$. Its complement is the universal cube.
            \item $f = 1$. Its complement is void.
            \item The cover $F$ consists of one implicant. Its complement is computed by
            De Morgan's law.
            \item The cover has a column of 0s, say in position $j$. Let $\alpha$ be a cube of all 1s, except
            in position $j$. Then $f = \alpha f_\alpha$, and $f' = a' + f'_\alpha$. Hence recursively compute
            the complement of $f_\alpha$, and of a and return their union. 
        \end{itemize}
    \end{solution*}
\end{columns}

\end{frame}

\begin{frame}[allowframebreaks]
\frametitle{Expand}
\begin{eg}
    Consider the function $f = a'b'c' + ab'c' + a'bc' + a'b'c$ with $abc'$ as
    a \textit{don't care} condition. Give the expanded cover.
\end{eg}

\newpage

\textbf{Solution} The set $F^{ON}$ in the positional-cube notation, is: $\begin{matrix}
10 & 10 & 10\\
01 & 10 & 10\\
10 & 01 & 10\\
10 & 10 & 01
\end{matrix}$. The set $F^{DC}$ is $01~01~10$. The set $F^{OFF}$ is then computed by complementation: $\begin{matrix}
01 & 11 & 01\\
11 & 01 & 01
\end{matrix}$. The column count vector is $(3, 1, 3, 1, 3, 1)$
and the weights of the implicants in $F$ are $(9,7,7, 7)$.
Hence, the second implicant is processed first, i.e., $01~10~10$.
The raising of the $0$ in column $1$ yields $11~10~10$,
which does not intersect the \textit{off set} and therefore is valid.
The raising of the $0$ in column $1$ yields $11~11~10$,
which is also valid,
but the raising of the $0$ in column $6$ would intersect the \textit{off set}
and it is rejected.

Hence the expanded implicant is $11~11~10$, which covers the first and third
implicants of the original cover $F^{ON}$
Thus $F$ can be updated to: $\begin{matrix}
11 & 11 & 10\\
10 & 10 & 01
\end{matrix}$.
The last implicant is processed next. A raise of the literal in column $2$ or $4$ would
make the implicant intersect the \textit{off set}. On the contrary, a raise in column $5$ is valid,
yielding $10~10~11$. The expanded cover is then:  $\begin{matrix}
11 & 11 & 10\\
10 & 10 & 11
\end{matrix}$ which is prime and minimal with respect to single-implicant containment.

\end{frame}

\begin{frame}[fragile]
\frametitle{Informal Reduce}
\begin{eg}
    Consider the function $f = c' + a'b'$. Let's try to reduce the first implicant.
\end{eg}

\onslide<2>\begin{solution*}
    The complement of the last implicant is: $\begin{matrix}
01 & 11 & 11\\
11 & 01 & 11
\end{matrix}$ whose intersection with the first implicant yields: $\begin{matrix}
01 & 11 & 10\\
11 & 01 & 10
\end{matrix}$.
Thus, the first implicant should be replaced by two cubes, which is not what we want.
By taking the supercube of the complement $11~11~11$,
and intersecting it with a we obtain it again, which shows us that it cannot be reduced.
\end{solution*}

\end{frame}

\begin{frame}[fragile]
\frametitle{Reduce Theorem}
\begin{theorem}
    Let $\alpha \in F$ be an implicant and $Q = F \cup F^{DC} - \{\alpha\}$.
    Then, the maximally reduced cube is $\tilde{\alpha}=\alpha \cap supercube(Q'_\alpha)$.
\end{theorem}

\onslide<2>\begin{proof}
    The maximally reduced cube can be computed by deleting from a those minterms
    that belong to $Q$ under the restriction that the result must be a single
    cube. Now $a \cap Q'$ can yield a set of cubes. Hence the maximally reduced
    cube is $\tilde{\alpha}=\alpha \cap supercube(Q')$),
    which is equivalent to $\tilde{\alpha}=\alpha \cap supercube(Q'_\alpha)$,
    because $Q'$ can be expanded
    as $(\alpha \cap Q'_\alpha) \cup (\alpha' \cap Q'_{\alpha'})$ and the second term has void
    intersection with $\alpha$.
    The cube is maximally reduced,
    because the supercube is the smallest cube containing $Q'$.
\end{proof}

\end{frame}

\begin{frame}[fragile]
\frametitle{Reduce Example}
\begin{eg}
    Consider the function $f = c' + a'b'$. Let's try to reduce the last implicant.
\end{eg}

\onslide<2>\begin{solution*}
    The maximally reduced cube is $\tilde{\alpha}=\alpha \cap supercube(Q'_\alpha)$,
    where $Q = F \cup F^{DC} - \{\alpha\}$, namely $11~11~10$.
    Hence $Q' = 11~11~01$, $Q'_\alpha = Q'$ and $supercube(Q'_\alpha) = Q'$.
    Then $\tilde{\alpha}=\alpha \cap Q' = 10~10~01$ and the reduced cover is: $\begin{matrix}
    11 & 11 & 10\\
    10 & 10 & 01
    \end{matrix}$.
\end{solution*}
\end{frame}

\begin{frame}[fragile]
\frametitle{Relatively Essential}
\begin{columns}
    \column{.6\linewidth}
    \begin{definition}
    The \textbf{relatively essential} set $E^r$, which contains those implicants that cover some
    minterms of the function not covered by other implicants of $F$.
\end{definition}

\begin{definition}
    The \textbf{totally redundant} set $R^t$: which contains those implicants that are covered by
    the relatively essential set.
\end{definition}

\begin{definition}
    The \textbf{partially redundant} set $R^p$, which contains the remaining implicants.
\end{definition}

\column{.5\linewidth}
\begin{eg}
    Consider the following cover: \begin{tabular}{c|ccc}
        & $a$ & $b$ & $c$\\ \midrule
        $\alpha$ & 10 & 10 & 11\\
        $\beta$ & 11 & 10 & 01\\
        $\gamma$ & 01 & 11 & 01\\
        $\delta$ & 01 & 01 & 11\\
        $\epsilon$ & 11 & 01 & 10
    \end{tabular}.
\end{eg}
\onslide<2>\begin{solution*}
    $E^r = \{\alpha, \epsilon\}$, $R^t = \emptyset$, $R^p = \{\beta, \gamma, \delta\}$.
\end{solution*}
\end{columns}
\end{frame}

\plain{Questions?}

\end{document}
