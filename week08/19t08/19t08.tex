%!TEX program = xelatex
\documentclass[10pt, compress, handout]{beamer}
\usepackage[titleprogressbar]{../../cls/beamerthemem}

\setbeamertemplate{caption}[numbered]
\setbeamertemplate{theorems}[numbered]
\newcounter{example}
\resetcounteronoverlays{example}
\newtheorem{crl}{Corollary}[theorem]
\newtheorem{eg}[example]{Example}
\newtheorem*{solution*}{Solution}

\usepackage{booktabs}
\usepackage[scale=2]{ccicons}
\usepackage{minted}

\usepackage{cleveref}
\crefname{example}{Example}{Examples}

\usepgfplotslibrary{dateplot}

\usemintedstyle{trac}

\usepackage{algorithm}
\usepackage[noend]{algpseudocode}
\resetcounteronoverlays{algorithm}

\usepackage{listings}
\usepackage{mathtools}
\usepackage{multicol}
\usepackage{qtree}

\usepackage{tikz}

\usepackage{version}
% \excludeversion{proof}
% \excludeversion{solution*}

\makeatletter
\def\old@comma{,}
\catcode`\,=13
\def,{%
    \ifmmode%
    \old@comma\discretionary{}{}{}%
    \else%
    \old@comma%
    \fi%
}
\makeatother

\title{CENG 4120 Tutorial of Week 08}
\subtitle{Multiple-Level Logic Optimization}
\author{LI Haocheng}
\institute{Department of Computer Science and Engineering}

\begin{document}

\maketitle

\begin{frame}
\frametitle{Kernel}
\begin{eg}
    \label{eg:kernel}
    Consider the kernel set of expression $f = ace + bce + de + g$.
\end{eg}
\onslide<2>\begin{solution*}
    Let us divide first $f$ by variable $a$, yielding $f = ce$. Since $ce$ is a single cube and not cube free, it is not a kernel. Similar considerations apply when dividing $f$ by $b$, $d$ or $g$ (single-cube
    quotient) and when dividing $f$ by $c$, yielding $ae + be$, which is not cube free.

    On the other hand, dividing $f$ by $e$ yields $ac + bc + d$, which is
    cube free. Hence $ac + bc + d$ is a kernel of $f$, and $e$ is the corresponding co-kernel.
    Similarly, dividing $f$ by $ce$ yields $a + b$, which is cube free. Thus $a + b$
    is a kernel of $f$ and $ce$ is the corresponding co-kernel. It is straightforward to verify
    that dividing $f$ by other cubes does not yield other kernels. Since $f$ is cube free, it is
    considered a kernel of itself, the corresponding co-kernel being $1$. Thus the kernel set of
    $f$ is $K(f) = \{(ac+ bc +d), (a + b), (ace + bce +de +g)\}$.
\end{solution*}
\end{frame}

\begin{frame}[allowframebreaks]
\frametitle{Brayton and McMullen's theorem}
\begin{theorem}
    Two expressions $f_a$ and $f_b$ have a common multiple-cube divisor $f_d$
    if and only if there exist kernels $k_a \in K(f_a)$ and $k_b \in K(f_b)$ such that $\left|k_a \cap k_b\right| \ge 2$.
\end{theorem}

\onslide<2>\begin{proof}
    Assume $\left|k_a \cap k_b\right| \ge 2$. Then the sum of two (or more) cubes in $k_a \cap k_b$ is
    necessarily a multiple-cube divisor of both $f_a$ and $f_b$, because any expression obtained
    by dropping some cubes from a divisor is also a divisor.
    Conversely, let $f_d$ be a multiple-cube expression dividing both $f_a$ and $f_b$. Since
    $f_d$ has more than one cube, there exists at least a cube-free expression $f_e$ that divides
    $f_d$.
    Therefore $f_e$ is a divisor of $f_a$. We claim that $f_e$ is a subset
    of the cubes of a kernel $k_a \in K(f_a)$. Indeed any divisor of $f_a$ is a subset of those cubes
    obtained by dividing $f_a$ by any cube of the corresponding quotient. In addition $f_e$ is cube
    free, and any expression obtained by adding cubes to $f_e$ has the same property. Therefore
    there exists a kernel $k_a \in K(f_a)$ containing $f_e$.
    By the same argument, $f_e$ is a subset of the cubes of a kernel
    $k_b \in K(f_b)$. Hence $f_e$ is a subset of the intersection of two kernels $k_a \cap k_b$ and it has
    two or more cubes, because it is cube free.
\end{proof}

\end{frame}

\begin{frame}[fragile]
\frametitle{Kernel Level}
\begin{columns}
    \column{.7\linewidth}
    \begin{eg}
        Consider the substitution of a network with the following expressions by kernel intersection:
        $f_x = ace+bce+de+g$, $f_y = ad+bd+cde+ge$, $f_z = abc$.
    \end{eg}
    
    \onslide<2->\begin{solution*}
        The kernels of $f_x$ were computed in~\Cref{eg:kernel} namely $K(f_x) = \{(a + b), (ac + bc + d), (ace + bce + de + g)\}$.
        The kernel set of $f_y$ is $K(f_y) = \{(a + b + ce), (cd + g), (ad + bd + cde + ge)\}$.
        The kernel set of $f_z$ is empty.
        There is only one kernel intersection, namely $a + b$, which can be extracted to yield:
        $f_w = a + b$, $f_x = wce + de + g$, $f_y = wd + cde + ge$, $f_z = abc$.
    \end{solution*}

\column{.4\linewidth}
\onslide<1->\begin{eg}
    Let $f_x = ace + bce + de + g$. Since it is cube free, $f_x = k_1 \in K(f_x)$.
    Expression $k_2 = ac + bc + d$ is a kernel of $f_x$.
    Expression $k_3 = a + b$ is a kernel of $k_2$. What are the levels of the kernels?
\end{eg}
\onslide<3>\begin{solution*}
    Kernel $k_3$ has no kernel but itself. Hence it has level $0$. Kernel $k_2$ has a level-$0$
    kernel and thus has level $1$, while kernel $k_1$ has a level-$1$ kernel and has level $2$.
\end{solution*}
\end{columns}
\end{frame}

\begin{frame}[fragile]
\frametitle{Find Kernel}
\begin{columns}
    \column{\linewidth}
    \onslide<1->\begin{eg}
        Consider again the kernel set of expression $f = ace + bce + de + g$ with the recursive algorithm.
    \end{eg}
    \onslide<2>\begin{solution*}
        No action is taken for $a$ and $b$.
        The third variable $c$ is in the support of two cubes. $ce$ is the first co-kernel being determined. The recursive call has as argument $a + b$.
        It does not find any additional kernel.
        Nevertheless, the recursive call adds a + b to the kernel set at the last step.
        No action is taken for $d$.
        $e$ is a co-kernel, and a recursive call is made with argument $ac + bc + d$.
        In the recursion,
        the kernel $a + b$ is rediscovered and added to the kernel set. Before returning from the
        recursion, $ac + bc + d$ is added to the kernel set.
        No action is taken for $g$.
        The last step adds $f = ace+bce+de+g$ to the kernel set.
        The algorithm returns $K = \{(ace + bce + de + g), (a + b), (ac + bc +d), (a + b)\}$.
    \end{solution*}
\end{columns}
\end{frame}

\begin{frame}[fragile]
\frametitle{Bookkeeping}
\begin{columns}
    \column{.8\linewidth}
    \onslide<1->\begin{eg}
        Give an algorithm to compute same kernels only once.
    \end{eg}
    \onslide<2>\begin{solution*}
        \begin{algorithm}[H]
            \caption{Find Kernel}
            \label{a-6}
            \begin{algorithmic}
                \Procedure{Find}{$f$, $i$}
                \State $K \coloneqq \emptyset$
                \For{$j \coloneqq i \cdots m$}
                \If{$\left|\Call{Cubes}{f, x_j}\right| \ge 2$} \textbf{continue}
                \EndIf
                \State $C \coloneqq \bigcap \Call{Cubes}{f, x_j}$
                \If{$x_k \notin C, \forall k < i$} $K \coloneqq K \cup \Call{Find}{f / C, j + 1}$
                \EndIf
                \EndFor
                \State \Return $K \cup \{f\}$
                \EndProcedure
            \end{algorithmic}
        \end{algorithm}
    \end{solution*}
\end{columns}
\end{frame}

\begin{frame}[fragile]
\frametitle{Find Unique Kernel}
\begin{columns}
    \column{\linewidth}
    \onslide<1->\begin{eg}
        Consider again the kernel set of expression $f = ace + bce + de + g$ with the improved recursive algorithm.
    \end{eg}
    \onslide<2>\begin{solution*}
        No action is taken for $a$ and $b$.
        The third variable $c$ is in the support of two cubes whose intersection $C = ce$.
        The recursive call has as arguments expression $a + b$ and pointer $j = 3 + 1$ without action.
        Nevertheless, it adds $a + b$ to the kernel set.

        No action is taken for $d$.
        The fifth variable $e$ is in the support of three cubes of $f$.
        A recursive call is done with arguments $ac + bc + d$ and pointer $j = 5 + 1$.
        In the recursion, only the sixth variable $g$ is considered and no kernel is discovered.
        Before returning from the recursion $ac + bc + d$ is added to the kernel set.
        No action is taken for $g$.
        The last step adds $f = ace + bce + de + g$ to the kernel set.
        The algorithm returns $K = \{(ace + bce + de + g), (ac + bc + d), (a + b)\}$.
    \end{solution*}
\end{columns}
\end{frame}

\begin{frame}[fragile]
\frametitle{Rectangle}
\begin{columns}
    \column{.4\linewidth}
    \onslide<1->\begin{definition}
        A \textbf{rectangle} of a matrix $A$ is a subset of rows and columns, where all
        entries are 1.
    \end{definition}
\begin{definition}
    The \textbf{co-rectangle} of a rectangle $(R, C)$ is the pair $(R, C')$, where $C'$
    is the complement of the column subset $C$.
\end{definition}
\column{.7\linewidth}
\begin{eg}
    Consider the matrix of expression $f_x = ace + bce + de + g$, with $m = 4$ cubes and
    $n = 6$ variables.
\end{eg}
    \onslide<2>\begin{solution*}
        \begin{table}
            \caption{Matrix}
            \label{tb:fx}
            \begin{tabular}{c|ccccccc}
            & var & $a$ & $b$ & $c$ & $d$ & $e$ & $g$\\\midrule
            cube & R \textbackslash C & $1$ & $2$ & $3$ & $4$ & $5$ & $6$  \\
            $ace$ & $1$ & $1$ & $0$ & $1$ & $0$ & $1$ & $0$\\
            $bce$ & $2$ & $0$ & $1$ & $1$ & $0$ & $1$ & $0$\\
            $de$ & $3$ & $0$ & $0$ & $0$ & $1$ & $1$ & $0$\\
            $g$ & $4$ & $0$ & $0$ & $0$ & $0$ & $0$ & $1$
        \end{tabular}
        \end{table}
    \end{solution*}
\end{columns}
\end{frame}

\begin{frame}[fragile]
\frametitle{Prime Rectangle}
\begin{columns}
    \column{.4\linewidth}
    \begin{eg}
        Find kernels from~\Cref{tb:fx}.
    \end{eg}
    \onslide<2->\begin{solution*}
        Prime rectangle $(\{1, 2\}, \{3, 5\})$ identifies co-kernel $ce$,
        whose kernel is identified by $(\{1, 2\}, \{1, 2, 4, 6\})$. Hence the kernel is $a + b$.

        The other prime rectangle $(\{1, 2, 3\}, \{5\})$ identities co-kernel $e$.
        The kernel is $ac + bc + d$.
    \end{solution*}
\column{.7\linewidth}
\onslide<1->\begin{eg}
    Extract cube intersection of expressions $f_x = ace + bce + de + g$ and $f_s = cde + b$.
\end{eg}
\onslide<3>\begin{solution*}
    \begin{tabular}{cc|ccccccc}
        & & var & $a$ & $b$ & $c$ & $d$ & $e$ & $g$\\\midrule
        cube & id & R \textbackslash C & $1$ & $2$ & $3$ & $4$ & $5$ & $6$  \\
        $ace$ & $x$ & $1$ & $1$ & $0$ & $1$ & $0$ & $1$ & $0$\\
        $bce$ & $x$ & $2$ & $0$ & $1$ & $1$ & $0$ & $1$ & $0$\\
        $de$ & $x$ & $3$ & $0$ & $0$ & $0$ & $1$ & $1$ & $0$\\
        $g$ & $x$ & $4$ & $0$ & $0$ & $0$ & $0$ & $0$ & $1$\\
        $cde$ & $s$ & $5$ & $0$ & $0$ & $1$ & $1$ & $1$ & $0$\\
        $b$ & $s$ & $6$ & $0$ & $1$ & $0$ & $0$ & $0$ & $0$\\
    \end{tabular}

A prime rectangle is $(\{1, 2, 5\}, \{3, 5\})$, corresponding to cube $ce$, which is present
in both expressions $f_x$ (row $1$, $2$) and $f_s$ (row $5$).
\end{solution*}
\end{columns}
\end{frame}

\plain{Questions?}

\end{document}
