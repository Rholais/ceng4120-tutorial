%!TEX program = xelatex
\documentclass[10pt, compress]{beamer}
\usepackage[titleprogressbar]{../../cls/beamerthemem}

\setbeamertemplate{caption}[numbered]
\setbeamertemplate{theorems}[numbered]
\newcounter{example}
\resetcounteronoverlays{example}
\newtheorem{crl}{Corollary}[theorem]
\newtheorem{eg}[example]{Example}
\newtheorem*{solution*}{Solution}

\usepackage{booktabs}
\usepackage[scale=2]{ccicons}
\usepackage{minted}

\usepackage{cleveref}
\crefname{example}{Example}{Examples}

\usepgfplotslibrary{dateplot}

\usemintedstyle{trac}

\usepackage{algorithm}
\usepackage[noend]{algpseudocode}
\resetcounteronoverlays{algorithm}

\usepackage{listings}
\usepackage{mathtools}
\usepackage{multicol}
\usepackage{qtree}

\usepackage{tikz}

\usepackage{version}
% \excludeversion{proof}
% \excludeversion{solution*}

\makeatletter
\def\old@comma{,}
\catcode`\,=13
\def,{%
    \ifmmode%
    \old@comma\discretionary{}{}{}%
    \else%
    \old@comma%
    \fi%
}
\makeatother

\title{CENG 4120 Tutorial of Week 03}
\subtitle{Generalized Cofactor}
\author{LI Haocheng}
\institute{Department of Computer Science and Engineering}

\begin{document}

\maketitle

\begin{frame}
\frametitle{Orthogonal Function}
\begin{definition}
    A set $\{\phi_1, \phi_2, \ldots, \phi_k\}$ of $n$-variable Boolean functions is called \textbf{orthogonal}
    provided \begin{equation}
        \label{eq:orthogonal}
        \phi_i \phi_j = 0, \forall i \ne j \in \{1, 2, \ldots, k\}.
    \end{equation}
\end{definition}

\begin{definition}
    A set $\{\phi_1, \phi_2, \ldots, \phi_k\}$ of $n$-variable Boolean functions is called \textbf{normal}
    provided \begin{equation}
    \label{eq:normal}
    \sum_{i = 1}^{k} \phi_i = 1.
    \end{equation}
\end{definition}

\begin{definition}
    A set $\{\phi_1, \phi_2, \ldots, \phi_k\}$ of $n$-variable Boolean functions is called \textbf{orthonormal}
    provided~\Cref{eq:orthogonal,eq:normal}.
\end{definition}
\end{frame}

\begin{frame}
\frametitle{Orthonormal Expansion}
\begin{definition}
    Let $\{\phi_1, \phi_2, \ldots, \phi_k\}$, be a orthonormal set.
    An \textbf{orthonormal expansion} of a function $f$ is \begin{equation}
        \label{eq:oe}
        f = \sum_{i = 1}^{k} f_{\phi_i} \phi_i.
    \end{equation}
    The term $f_{\phi_i}$ is called the \textbf{generalized cofactor} of $f$ w.r.t. $\phi_i, \forall i$.
    
    Note that in the special case on a
    variable $x_i$, the orthonormal functions are just the literals $x_i$ and $x_i'$
    and the generalized cofactor is the usual cofactor.
\end{definition}
\end{frame}

\begin{frame}[allowframebreaks]
\frametitle{Generalized Cofactor}
\begin{theorem}
    Let $\{\phi_1, \phi_2, \ldots, \phi_k\}$, be a orthonormal set and let $f$
    be a Boolean function. Then $f$ is given by~\Cref{eq:oe} if and only
    if \begin{equation}
    \label{eq:gc}
    f_{\phi_i} \phi_i = f \phi_i \qquad (\forall i \in \{1, \ldots, k\}).
    \end{equation}
\end{theorem}
\newpage
\textbf{Proof.} Suppose~\Cref{eq:oe} to be valid. Then, for any element $\phi_j \in \{\phi_1, \phi_2, \ldots, \phi_k\}$,
\begin{equation}
    f \phi_j = \sum_{i = 1}^{k} f_{\phi_i} \phi_i \phi_j.
\end{equation}
The set $\{\phi_1, \phi_2, \ldots, \phi_k\}$ satisfies the orthogonal-condition~\Cref{eq:orthogonal};
hence, \begin{equation}
    f \phi_j = f_{\phi_j} \phi_j,
\end{equation}
verifying condition~\Cref{eq:gc}. Suppose on the other hand that~\Cref{eq:gc} holds. Then,
\begin{equation}
    \sum_{i = 1}^{k} f_{\phi_i} \phi_i = \sum_{i = 1}^{k} f \phi_i = f \sum_{i = 1}^{k} \phi_i = f.
\end{equation}
\end{frame}

\begin{frame}
\frametitle{Design Synthesis}
\begin{itemize}
    \item Register-transfer level (RTL): digital signals and logical operations.
    \item Netlist: electronic components and connections.
    \item Synopsys Design Compiler (DC): RTL model to gate-level netlist.
    \item \texttt{.db} standard-cell library: logic and timing
\end{itemize}

\begin{itemize}
    % \item \texttt{.cdl} circuit design language: circuit schematics for each cell
    \item \texttt{.gds} graphic database system: layout for each cell
    \item \texttt{.v} Verilog: behavioral specification for each cell
    \item \texttt{.sp} schematics with extracted parasitics for each cell
    \item \texttt{.lib} liberty: abstract logical, timing, power view for each cell
    \item \texttt{.lef} library exchange format: abstract physical view for each cell
    \item \texttt{.db} binary compiled version of \texttt{.lib} file
\end{itemize}
\end{frame}

\begin{frame}[fragile]
\frametitle{Verilog}
\begin{columns}
    \column{.6\linewidth}
    \lstset{language=Verilog,
        basicstyle=\ttfamily\scriptsize
    }
\begin{lstlisting}
module NAND2X1 (A, B, Y);
input  A ;
input  B ;
output Y ;
    and (I0_out, A, B);
    not (Y, I0_out);
    
    specify
        // delay parameters
        specparam
            tplhl$A$Y = 0.2:0.2:0.2,
            tphlh$A$Y = 0.35:0.35:0.35,
            tplhl$B$Y = 0.2:0.2:0.2,
            tphlh$B$Y = 0.34:0.34:0.34;
        // path delays
        (A *> Y) = (tphlh$A$Y, tplhl$A$Y);
        (B *> Y) = (tphlh$B$Y, tplhl$B$Y);
    endspecify
endmodule
\end{lstlisting}

\column{.4\linewidth}
\begin{itemize}
    \item Verilog primitive gates
    \item Delay
\end{itemize}
\end{columns}

\end{frame}

\begin{frame}[fragile]
\frametitle{Extracted Parasitics}
\begin{columns}
    \column{.6\linewidth}
    \lstset{language=Verilog,
        basicstyle=\ttfamily\scriptsize
    }
    \begin{lstlisting}
.subckt NAND2X1 vdd Y gnd A B
M0 Y A vdd vdd pmos w=0.5u l=0.05u
+ ad=0p pd=0u as=0p ps=0u
M1 vdd B Y vdd pmos w=0.5u l=0.05u
+ ad=0p pd=0u as=0p ps=0u
M2 a_9_6# A gnd gnd nmos w=0.5u l=0.05u
+ ad=0p pd=0u as=0p ps=0u
M3 Y B a_9_6# gnd nmos w=0.5u l=0.05u
+ ad=0p pd=0u as=0p ps=0u
.ends NAND2X1
    \end{lstlisting}
    
    \column{.4\linewidth}
    \begin{itemize}
        \item Resistance
        \item Capacitance
    \end{itemize}
\end{columns}

\end{frame}

\begin{frame}[fragile]
\frametitle{Liberty}
\begin{columns}
    \column{.6\linewidth}
    \lstset{language=Verilog,
        basicstyle=\ttfamily\scriptsize
    }
    \begin{lstlisting}
cell (NAND2X1) {
    area : 1.877200;
    cell_leakage_power : 3.29145;
    pin(A)  {
        direction : input;
        capacitance : 0.00211057;
        rise_capacitance : 0.00211057;
        fall_capacitance : 0.00188435;
    }
    ...
    pin(Y)  {
        direction : output;
        ...
        function : "(!(A B))";
        timing() {
            related_pin : "A";
            timing_sense : negative_unate;
            cell_rise(delay_template_6x6) {
                index_1 ("0.1, 0.5, 1.2, 3, 4, 5");
                index_2 ("0.06, 0.24, 0.48, 0.9, 1.2, 1.8");
                values ( \
                "0.345107, 0.400669, 0.461807, 0.611732, 0.650215, 0.820245", \
                ...);
            }
    \end{lstlisting}
    
    \column{.4\linewidth}
    \vspace{-1in}
    \begin{itemize}
        \item Area
        \item Leakage power
        \item Capacitance of each input pin
        \item Internal power
        \item Logical functionality
        \item Timing
    \end{itemize}
\end{columns}

\end{frame}

\begin{frame}[fragile]
\frametitle{Library Exchange Format}
\begin{columns}
    \column{.6\linewidth}
    \lstset{language=Verilog,
        basicstyle=\ttfamily\scriptsize
    }
    \begin{lstlisting}
MACRO NAND2X1
    CLASS CORE ;
    ORIGIN 0 0 ;
    SIZE 0.76 BY 2.47 ;
    SYMMETRY X Y ;
    SITE CoreSite ;
    PIN A
        DIRECTION INPUT ;
        USE SIGNAL ;
        PORT
            LAYER metal1 ;
                RECT 0.045 0.7725 0.235 0.9825 ;
        END
    END A
    ...
    PIN gnd
        DIRECTION INOUT ;
        USE GROUND ;
        PORT
            LAYER metal1 ;
                RECT 0.16 -0.065 0.225 0.7075 ;
                RECT 0 -0.065 0.76 0.065 ;
        END
    END gnd
    \end{lstlisting}
    
    \column{.4\linewidth}
    \vspace{-1in}
    \begin{itemize}
        \item Dimension
        \item Signal pins
        \item Power/ground pins
    \end{itemize}
\end{columns}

\end{frame}

\begin{frame}[fragile]
\frametitle{Set Variables}
\begin{columns}
    \column{\linewidth}
    \lstset{language=tcl,
        basicstyle=\ttfamily\scriptsize
    }
    \begin{lstlisting}
set OSU_FREEPDK [format "%s%s" [getenv "PDK_DIR"] "/osu_soc/lib/files"]
set target_library "$OSU_FREEPDK/gscl45nm.db"
set link_library "$OSU_FREEPDK/gscl45nm.db"
    \end{lstlisting}
    
    % \column{.4\linewidth}
    \begin{itemize}
        \item \texttt{PDK\_DIR}: process design kit directory
        \item \texttt{OSU\_FREEPDK}: library file directory
        \item \texttt{target\_library}: list of technology libraries of components to be used when compiling a design
        \item \texttt{link\_library}: list of design files and libraries used during linking
    \end{itemize}
\end{columns}

\end{frame}

\begin{frame}[fragile]
\frametitle{Analyze Files}
\begin{columns}
    \column{\linewidth}
    \lstset{language=tcl,
        basicstyle=\ttfamily\scriptsize
    }
    \begin{lstlisting}
analyze -f verilog $my_verilog_files
    \end{lstlisting}
    
    % \column{.4\linewidth}
    \begin{itemize}
        \item Analyzes  the  specified  HDL source files and stores the design templates they define into the specified  library  in  a  format ready  to  specialize  and elaborate to form linkable cells of a full design.
    \end{itemize}

    \begin{lstlisting}
Running PRESTO HDLC
Searching for ./basic_components.v
Searching for ./cla16.v
Searching for ./multi.v
Searching for ./multiplyadd.v
Compiling source file ./basic_components.v
Compiling source file ./cla16.v
Compiling source file ./multi.v
Compiling source file ./multiplyadd.v
Presto compilation completed successfully.
Loading db file '/research/fyyoung/hcli/pl/NCSU-FreePDK45-1.4/FreePDK45/osu_soc/lib/files/gscl45nm.db'
    \end{lstlisting}
\end{columns}

\end{frame}

\begin{frame}[fragile]
\frametitle{Elaborate Design}
\begin{columns}
    \column{\linewidth}
    \lstset{language=tcl,
        basicstyle=\ttfamily\scriptsize
    }
    \begin{lstlisting}
elaborate $my_toplevel
    \end{lstlisting}
    
    % \column{.4\linewidth}
    \begin{itemize}
        \item Builds  a  design from the intermediate format of a Verilog module, a VHDL entity and architecture, or a VHDL configuration.
    \end{itemize}
    
    \begin{lstlisting}
Loading db file '/opt2/synopsys/syn-J-2014.09-SP5-6/libraries/syn/gtech.db'
Loading db file '/opt2/synopsys/syn-J-2014.09-SP5-6/libraries/syn/standard.sldb'
Loading link library 'gscl45nm'
Loading link library 'gtech'
Running PRESTO HDLC
Presto compilation completed successfully.
Elaborated 1 design.
Current design is now 'multiplyadd'.
Information: Building the design 'multi'. (HDL-193)
Presto compilation completed successfully.
...
    \end{lstlisting}
\end{columns}

\end{frame}

\begin{frame}[fragile]
\frametitle{High-Effort Compile}
\begin{columns}
    \column{\linewidth}
    \lstset{language=tcl,
        basicstyle=\ttfamily\scriptsize
    }
    \begin{lstlisting}
set my_period [expr 1000 / $my_clk_freq_MHz]
create_clock -period $my_period $my_clock_pin

compile_ultra -incremental
    \end{lstlisting}
    
    % \column{.4\linewidth}
    \begin{itemize}
        \item Creates  a  clock object and defines its waveform in the current design.
        \item Performs  a high-effort compile on the current design for better quality of results (QoR).
    \end{itemize}
\end{columns}

\end{frame}

\begin{frame}[fragile]
\frametitle{Write File}
\begin{columns}
    \column{\linewidth}
    \lstset{language=tcl,
        basicstyle=\ttfamily\scriptsize
    }
    \begin{lstlisting}
set filename [format "%s%s"  $my_toplevel ".vhh"]
write_file -f verilog -hierarchy -output $filename
    \end{lstlisting}
    
    % \column{.4\linewidth}
    \begin{itemize}
        \item Writes a design netlist or schematic from memory to a file.
    \end{itemize}
\end{columns}

\end{frame}

\plain{Questions?}

\end{document}
