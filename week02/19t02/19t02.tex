%!TEX program = xelatex
\documentclass[10pt, compress, handout]{beamer}
\usepackage[titleprogressbar]{../../cls/beamerthemem}

\setbeamertemplate{caption}[numbered]
\setbeamertemplate{theorems}[numbered]
\newcounter{example}
\resetcounteronoverlays{example}
\newtheorem{crl}{Corollary}[theorem]
\newtheorem{eg}[example]{Example}
\newtheorem*{solution*}{Solution}

\usepackage{booktabs}
\usepackage[scale=2]{ccicons}
\usepackage{minted}

\usepackage{cleveref}
\crefname{example}{Example}{Examples}

\usepgfplotslibrary{dateplot}

\usemintedstyle{trac}

\usepackage{algorithm}
\usepackage[noend]{algpseudocode}
\resetcounteronoverlays{algorithm}

\usepackage{version}
% \excludeversion{proof}
% \excludeversion{solution*}

\usepackage{mathtools}
\usepackage{multicol}
\usepackage{qtree}

\usepackage{tikz}

\makeatletter
\def\old@comma{,}
\catcode`\,=13
\def,{%
    \ifmmode%
    \old@comma\discretionary{}{}{}%
    \else%
    \old@comma%
    \fi%
}
\makeatother

\title{CENG 4120 Tutorial of Week 02}
\subtitle{Boolean Functions}
\author{LI Haocheng}
\institute{Department of Computer Science and Engineering}

\begin{document}

\maketitle

\subsection{Shannon Expansion}

\begin{frame}[allowframebreaks]
\frametitle{Shannon Expansion}
\begin{theorem}
	\label{thm:shannon}
    If $f: \mathbb{B}^n \rightarrow \mathbb{B}$ is a Boolean function, then
    \begin{equation}
    	\forall (x_1, \ldots) \in \mathbb{B}^n,
    	f(x_1, \ldots) = x_1' f(0, \ldots) + x_1 f(1, \ldots).
    \end{equation} 
\end{theorem}
\newpage

\textbf{Proof.} \begin{enumerate}
	\item Suppose $f$ is a \textit{constant function}, defined by $f(X) = b$ for all $X \in \mathbb{B}^n$,
	where $b$ is a fixed element of $\mathbb{B}$. Then \begin{equation}
		f(x_1, \ldots) = b = (x_1' + x_1) b = x_1'b + x_1b = x_1' f(0, \ldots) + x_1 f(1, \ldots).
	\end{equation}
	\item Suppose $f$ is a \textit{projection function}, defined by $f(X) = x_1$ for all $X \in \mathbb{B}^n$. Then \begin{equation}
	f(x_1, \ldots) = x_1 = 0 + x_1 \cdot 1 = x_1' \cdot 0 + x_1 \cdot 1 = x_1' f(0, \ldots) + x_1 f(1, \ldots).
	\end{equation}
	\item Suppose $f$ is a \textit{projection function}, defined by $f(X) = x_i$ for all $X \in \mathbb{B}^n$, where $i \in \{2, \ldots, n\}$. Then \begin{equation}
	f(x_1, \ldots) = x_i = (x_1' + x_1) x_i = x_1' x_i + x_1 x_i = x_1' f(0, \ldots) + x_1 f(1, \ldots).
	\end{equation}
	\newpage
	
	\item Suppose~\Cref{thm:shannon} to hold for Boolean functions $g$ and $h$, i.e., \begin{align}
		\begin{aligned}
		g(x_1, \ldots) & = x_1' g(0, \ldots) + x_1 g(1, \ldots),\\
		h(x_1, \ldots) & = x_1' h(0, \ldots) + x_1 h(1, \ldots).
		\end{aligned}
	\end{align}
	We show that~\Cref{thm:shannon} then holds
	for the $n$-variable functions $g + h$, $gh$, and $g'$.
	\begin{enumerate}
		\item \begin{align}
			\begin{aligned}
			(g + h)(x_1, \ldots) = & g(x_1, \ldots) + h(x_1, \ldots)\\
			= & (x_1' g(0, \ldots) + x_1g(1, \ldots)) + (x_1'h(0, \ldots) + x_1h(1, \ldots))\\
			= & x_1'(g(0, \ldots) + h(0, \ldots)) + x_1(g(1, \ldots) + h(1, \ldots))\\
			= & x_1'[(g + h)(0, \ldots)] + x_1[(g + h)(1, \ldots)]. 
			\end{aligned}
		\end{align}
		\item \begin{align}
			\begin{aligned}
			(gh)(x_1, \ldots) = & (g(x_1, \ldots))(h(x_1, \ldots))\\
			= & (x_1'g(0, \ldots) + x_1g(1, \ldots)) (x_1'h(0, \ldots) + x_1h(1, \ldots))\\
			= & x_1' (g(0, \ldots)h(0, \ldots)) + x_1(g(1, \ldots)h(1, \ldots))\\
			= & x_1'[(gh)(0, \ldots)] + x_1[(gh)(1, \ldots)].
			\end{aligned}
		\end{align}
		\item \begin{align}
			\begin{aligned}
			(g')(x_1, \ldots) = & (g(x_1, \ldots))'\\
			= & (x_1'g(0, \ldots) + x_1g(1, \ldots))'\\
			= & (x_1 + (g(0, \ldots))')(x_1' + (g(1, \ldots))')\\
			= & x_1'[(g')(0, \ldots)] + x_1[(g')(1, \ldots)] + [(g')(0, \ldots)][(g')(1, \ldots)]\\
			= & x_1'[(g')(0, \ldots)] + x_1[(g')(1, \ldots)]. 
			\end{aligned}
		\end{align}
	\end{enumerate}
\end{enumerate}
\end{frame}

\begin{frame}[fragile]
\frametitle{Min-term}
\begin{definition}
	Any function can be expressed as a \textbf{sum of products} of $n$ literals, called the
	\textbf{minterms} of the function, by recursive expansion. Alternatively, it can be represented
	as a \textbf{product of sums} of $n$ literals. called \textbf{maxterms}.
\end{definition}
\begin{columns}
	\column{.8\linewidth}
	\begin{eg}
		\label{eg:cofactor}
		Consider the majority function $f = ab + bc + ac$. \begin{enumerate}
			\item Write out its cofactor w.r.t. $a$ and $a'$.
			\item Re-write the function with the cofactors above.
			\item Write out its cofactor w.r.t. $ab$, $ab'$, $a'b$, and $a'b'$.
			\item Re-write the function with the cofactors above.
		\end{enumerate}
	\end{eg}
	\column{.3\linewidth}
	\begin{solution*}
		\begin{enumerate}
			\item<2-> $f_{a} = b + c, f_{a'} = bc$.
			\item<3-> $f = a(b + c) + a'bc$.
			\item<4-> $f_{ab} = 1, f_{ab'} = c, f_{a'b} = c, f_{a'b'} = 0$.
			\item<5> $f = ab + ab'c + a'bc$.
		\end{enumerate}
	\end{solution*}
\end{columns}
\end{frame}

\begin{frame}[fragile]
\frametitle{Unate}
\begin{definition}
    A function $f(x_1, x_2, \ldots, x_i, \ldots, x_n)$ is (positive/negative) \textbf{unate in variable} $x_i$, if $f_{x_i} \supseteq f_{x_i'} (f_{x_i} \subseteq f_{x_i'})$. Otherwise it is binate (or mixed) in that variable.
    
    A function is (positive/negative) \textbf{unate} if it is (positive/negative) unate in all
    support variables. Otherwise it is binate (or mixed). 
    
    Note that stating $f_{x_i} \supseteq f_{x_i'}$ is equivalent to say that the set of minterms of $f_{x_i}$ includes the set of minterms of $f_{x_i'}$, or $f_{x_i} \ge f_{x_i'}$, for all possible assignments to the
    other variables $x_j \colon j \ne i, j = 1, 2, \ldots, n$. 
\end{definition}
\begin{eg}
	Is $f = a + b + c'$ is positive or negative unate w.r.t. variable $a$, $b$, or $c$?
\end{eg}
\onslide<2>\begin{solution*}
	$f$ is positive unate w.r.t. variable $a$ and $b$.
	$f$ is negative unate w.r.t. variable $c$.
\end{solution*}
\end{frame}

\begin{frame}[fragile]
\frametitle{Quantification}
\begin{eg}
    Consider the majority function $f = ab + bc + ac$ with $f_{a} = b + c$ and $f_{a'} = bc$ in~\Cref{eg:cofactor}. Write out \begin{enumerate}
    	\item the Boolean difference w.r.t. $a$,
    	\item the universal quantification of $f$ w.r.t. $a$, and
    	\item the existential quantification of $f$ w.r.t. $a$.
    \end{enumerate}
\end{eg}
\begin{solution*}
    \begin{enumerate}
        \item<2-> $\frac{\partial f}{\partial a} = bc' + b'c$.
        \item<3-> $(\forall a~f) = bc$.
        \item<4> $(\exists a~f) = b + c$.
    \end{enumerate}
\end{solution*}
\end{frame}

\plain{Questions?}

\end{document}
